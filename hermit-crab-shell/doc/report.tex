\documentclass{article}

\usepackage[slantfont, boldfont]{xeCJK}
\usepackage{fancyhdr}
\usepackage{graphicx}
\usepackage{geometry}
\usepackage{indentfirst}
\usepackage{mdwlist}
\usepackage{listings}

% set fonts
\setCJKmainfont[BoldFont={Adobe Heiti Std},
                ItalicFont={Adobe Kaiti Std}]
               {Adobe Song Std}
\setCJKmonofont{Adobe Fangsong Std}
%\setmainfont[Mapping=tex-text]{Times New Roman}
%\setsansfont{Helvetica}
%\setmonofont{Courier New}
\punctstyle{kaiming}

% set page style
\pagestyle{fancy}
\fancyhf{}
\lhead{\textbf{Project 1: Hermit Crab Shell}}
\rhead{\leftmark}
\cfoot{\thepage}
\renewcommand{\headrulewidth}{1pt}
\renewcommand{\footrulewidth}{0pt}

% set figure, table, etc name
\renewcommand\figurename{图}
\renewcommand\tablename{表}
\renewcommand\contentsname{目录}

% set listing style
\lstset{numbers=none,
       numberstyle=\scriptsize,
       frame=lines,
       flexiblecolumns=false,
       language=,
       basicstyle=\ttfamily\small,
       breaklines=true,
       extendedchars=true,
       escapechar=\%,
       texcl=true,
       showstringspaces=true,
       keywordstyle=\bfseries,
       tabsize=4}

% set title page
\title{\textbf{操作系统Project 1报告}\\(Hermit Crab Shell)}
\author{隋清宇\\5090309011}
\date{2011-11}

\begin{document}
\maketitle
\thispagestyle{empty}
\newpage

\tableofcontents
\thispagestyle{empty}
\newpage

\setcounter{page}{1}
\section{概述}
本项目名为Hermit Crab Shell,目的是实现一个简单的shell。

本程序可以在Minix下编译运行,也可以在Linux下编译运行;需要readline库的支持。详细编译参数请参照makefile。

\section{项目结构}
本项目包含多个文件,每个文件的作用如下:

\subsection{shell.c}
本文件为shell的启动文件,包含main函数。其中包括了shell运行的整个流程:读入命令——处理命令——执行命令。

\subsection{const.h}
本文件中定义了shell所需要的几个全局常量,包括一条命令的最大长度等。可以于编译前修改此文件以更改这些设置。

\subsection{debug.h}
这个头文件在每个文件前都会被包括,作用是定义一个DEBUG宏,以开启一些调试信息。

\subsection{dir.c/h}
这两个文件中包含了一些和目录处理有关的函数。

\subsection{inline.c/h}
这两个文件中包含了内部命令的具体实现。

\subsection{command.c/h}
这两个文件中定义了一系列和命令相关的数据结构,并实现了运行一条命令的函数。

\subsection{prompt.c/h}
这两个文件中定义了一些和获取当前命令提示符相关的函数。

\subsection{tokens.c/h}
这两个文件中定义了一系列和token相关的数据结构,以及它们的构造和析构函数(需要手动调用)。

\subsection{lexer.lex/h}
这两个文件是词法分析器的实现,使用Flex生成。

\subsection{parser.c/h}
这两个文件是语法分析器的实现,实现中手动遍历了语法树,并提取出相关信息传递给command.c或inline.c中的函数执行。

\subsection{makefile}
这是整个项目的makefile,可以使用make编译。

\section{项目功能}

\subsection{基本界面}
本shell由显示当前路径的输入提示符开头,可以在之后输入命令的内容。

\subsection{执行可执行文件}
可以执行包括参数的可执行文件。

\subsection{简单的错误提示}
如果出现语法错误会报错并提示错误号。

错误号的具体对应详见parser.h内的枚举类型PARSE\_STATUS。

\subsection{内置简单shell命令}
内置有cd和exit命令,分别用来更改当前路径与退出shell。

exit命令支持一个正整数作为参数,表示退出时的错误号。

\subsection{I/O重定向}
使用>、<或>$ $>符号进行I/O重定向。

\subsection{管道}
使用|符号进行无名管道连接。

\subsection{后台运行}
使用\&符号使一个程序后台运行。

\subsection{顺序执行多个命令}
使用;符号顺序执行多个命令。

\subsection{多个重定向、管道、后台运行与执行命令的组合}
以上几种连接符可以交替使用,运行效果与bash一致。

\subsection{历史记录与自动补全}
使用readline和history库实现,与bash的用法基本一致。使用上、下键可以切换历史记录,使用tab键进行补全。

\subsection{字符串支持}
支持双引号字符串,连接方式与bash类似,实现了一部分通配符。

例如:执行命令

\begin{lstlisting}
e"ch"o "abc""def"
\end{lstlisting}

可以输出字符串``abcdef''(不含引号)。

\section{设计思路}
最开始,我试图不使用词法和语法分析器完成这个shell。但是很可惜,随着代码的编写,我发现这样有很多扩展将很难完成。于是我更改了程序的架构,使用Flex生成了词法分析器,并自己编写了语法分析器,从而使程序的可扩展性上升了一个级别。程序也是从这时开始被提交到GitHub维护的。

整个程序的大致运行过程为:

\begin{enumerate*}
\item 显示提示符并读入命令;
\item 对命令进行词法分析得到token流;
\item 对token流进行语法分析得到command;
\item 调用函数执行command;
\item 在command全部执行后进行waitpid等操作;
\item 重复这个过程。
\end{enumerate*}

\section{设计细节}

\subsection{自动补全和历史记录}
这一部分的完成上,我使用了readline库。在控制历史记录的时候,出现了一些小麻烦——运行过程中有很多历史记录变成了乱码。

经过调试,我发现add\_history这个函数并不会自动拷贝传入的字符串到新的内存。所以我将代码改为不free传入的内存,成功。(在注释中有说明:``do NOT free single\_line!'')

\subsection{词法分析和语法分析}
词法分析器的部分,我是用Flex完成的。但是即使这样,还是遇到了不小的麻烦。

因为编译原理的大作业我是使用ANTLR完成的,所以对Flex和yacc并不熟悉。一开始的时候,我完全是在“摸着石头过河”。而这时我还没有查询Flex和yacc怎么连接,这直接导致了我的token符号定义是用了自己的tokens.h文件。

而在查阅了更多资料之后,我发现如果想要让Flex和yacc连接,必须使用yacc生成的tokens文件,这样要对代码做不小的改动。于是,我决定自己手写一个语法分析器。

语法分析器的实现中,遇到的问题并不大。我设计了几个函数,每个函数可以分析一个特定的语法单元,并调用其它的函数进行进一步处理。

在最开始我有一个做法是,每分析出一个command随即在parser中执行。但是这样做实际上会在程序规模扩大后导致增加新功能的花费增多,并不是一个很好的做法。所以,我在之后的版本中将command抽象成了一个数据结构,这样就可以用一个更普适的处理函数去处理执行命令的步骤,也增强了代码的内聚性。

\subsection{内存泄露}
调试代码的时候,我遇到的最大问题就是严重的内存泄漏。因为之前很少用C编写代码,突然从Python和C\#等有内存管理的语言转到C语言,有一些不太熟悉。

直到我找到了一个叫做valgrind的工具,这个问题才得以解决。这个工具可以检查C/C++程序中的内存泄露并报错,非常适合调试程序。

但实际上,一些更好的编程风格会使得内存泄露问题不再出现的那么频繁。比如将大部分对象定义为指针类型,并为它们提供构造和析构函数;拷贝对象的时候提供函数,在内部进行深拷贝等。

\section{其它信息}

\subsection{最新版本}
可以在GitHub上获取到本程序的最新版本。

地址:https://github.com/sqybi/OS-Homework/tree/master/Chapter2/p02

\subsection{协议}
所有的代码都在以下协议下发布:

GNU General Public License v3:http://www.gnu.org/licenses/gpl.html

\end{document}